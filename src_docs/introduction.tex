%!TEX root=introduction.tex
\documentclass[utf8, 12pt, a4paper, oneside]{ctexart}
\usepackage{amsmath, amsthm, amssymb, graphicx}
\usepackage[bookmarks=true, colorlinks, citecolor=blue, linkcolor=black]{hyperref}
\title{Lesson0:Introduction}
\author{Chen}
\date{\today}


\begin{document}

\maketitle

\newpage

\section{为什么是Python?}
\begin{itemize}
    \item 你后续的Kaggle:房价预测会用到
    \item 你后续的高阶物理会用到
    \item Python是一项任何理工科学生都会用到的超好用工具,早会早爽
\end{itemize}

\newpage


\section{为什么不是C、C++、Rust、golang、java、javascript...}
\begin{itemize}
    \item 你没有太多时间去钻研一门复杂的语言/了解计算机底层的实现
    \item 你暂时没有真正意义上的开发需求
    \item 你并不面临着处理复杂系统或是与性能相关的领域
    \item 学习成本和编码效率对你而言比程序本身的运行效率更重要
\end{itemize}

\newpage


\section{Python的优点与特性}
\begin{itemize}
    \item Python是一门远离底层、高度抽象的语言,允许你把精力都放在对现实逻辑的抽象上,而不需要考虑一些细碎繁琐的问题。你只需要考虑如何把现实问题翻译为数据与函数,而不必过分关注数据被怎样储存、变量的生命周期和所有权等复杂问题。
    \item Python的语法相当简洁且符合直觉,你应该能在很短时间内掌握
    \item Python在各种领域都有相当丰富的库/框架,尤其是在你目前关心的数据科学领域,你不需要从头实现一些东西
    \item Python能做的事情远远超出我将为你介绍的,在未来的日子里你可以用它做几乎任何你感兴趣的事情,包括但不限于图形渲染、前后端、模型训练...
\end{itemize}

\newpage


\section{你将在课程中学到}
\begin{itemize}
    \item Python的基本语法:我并不是一个编程语言爱好者,所以我不会向你介绍“回字的四种写法”,我只会向你介绍使用频率最高的特性,用于支持你学习后续的课程
    \item 如何对现实逻辑进行抽象:让现实逻辑抽象为数据、函数与类的交互
    \item Matplotlib/seaborn/pandas/numpy等常用的数据处理库中你可能需要的部分
\end{itemize}

\newpage


\section{在课程正式开始前,你可以先准备的}
\begin{itemize}
    \item 安装Python 3.7或以上版本 \href{https://www.python.org/}{Python官网} \href{https://zhuanlan.zhihu.com/p/439514720}{Python安装参考教程}
    \item 安装vscode \href{https://zhuanlan.zhihu.com/p/364910894}{vscode安装参考教程}
    \item 安装pip  \href{https://zhuanlan.zhihu.com/p/38603105}{pip安装参考教程}
\end{itemize}
安装过程中有任何疑问,随时联系我

\newpage


\section{暑期课程架构}
\begin{enumerate}
    \item Python基础语法
    \item 函数专题
    \item 面向对象专题
    \item 其他特性与代码组织
    \item 数据科学常用库简介
    \item 机器学习入门知识指路    
\end{enumerate}

\newpage


\section{近期课程安排}
Python基础语法(预计4-6h完成)
\begin{enumerate}
    \item 引入 + 程序的基本构造
    \item 数据类型(1) + 基础IO
    \item 程序的控制结构(1):判断
    \item 数据类型(2)
    \item 程序的控制结构(2):循环
    \item 复习(1) + 习题(1)
\end{enumerate}


\end{document}